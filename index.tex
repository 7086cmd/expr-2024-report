\documentclass{article}
\usepackage{amsmath, amssymb, amsthm}
\usepackage[UTF8]{ctex}
\usepackage{graphicx}
\usepackage{subfigure}
\usepackage{float}
\usepackage{geometry}
\usepackage{fancyhdr}
\usepackage{hyperref}
\usepackage{fontspec}
\fontspec{Times New Roman}
\usepackage{enumitem}

\title{``魔板'' 加持下的平抛运动定理验证实验}\

\date{\today}

\begin{document}

\maketitle

\abstract{
    通过朗威 (DISLab) 的 ``魔板'' 系统, 可以即时获取平抛物体的运动轨迹. 通过以时间, $x$ 轴坐标和 $y$ 轴坐标为变量的数据, 可以验证平抛运动的各项定理. 本实验通过 ``魔板'' 系统, 以及 ``魔板'' 系统配套的实验仪器, 验证了平抛运动的各项定理.

    在中学课本中, 平抛运动的定理的验证是通过复写纸, 通过速度的分解来分别得出数值和平行方向的运动特点, 进而验证平抛运动的模型. 然而, 传统实验中, 初始防止小球的鞋面需要人用受来控制小球的释, daozhi shifang
}

\tableofcontents
\end{document}